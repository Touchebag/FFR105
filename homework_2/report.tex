\documentclass{article}

\usepackage{fancyhdr}
\usepackage{mathtools}

\pagestyle{fancy}

\lhead{Robin Touche 900610-3270}
\rhead{robint@student.chalmers.se}

\begin{document}
\section*{Home problem 2}

\setcounter{section}{2}
\subsection{}
\subsubsection*{a}

The total number of permutations is $N!$. To remove shifted version we have to
divide this number by $N$ since each cycle can start at any of the $N$ cities.
Then we need to further divide by $2$ to remove mirrored paths as well.

The final answer is thus $\frac{N!}{2N}$.

\subsubsection*{b,c,d}

Note that to be able to run any of the srcipts TSPgraphics needs to be added to
the path. Also, \emph{b} and \emph{d} have many conflicting names so make sure
to not load them at the same time.

Also of note is that I pre-calculate the distances between all cities and pass
that matrix around (in the function arguments in \emph{b,d} and by a global
matrix, unfortunately, in \emph{d}) to avoid having to re-calculate them every
step of the way.  This gave a speedup of up to 10x on some of the systems I
tested.

The shortest path length for each algorithm can be seen in the table below. The
parameters used in these cases have been left in the code file.

\begin{center}
  \begin{tabular}{| c | c |}
    \hline
    GA & value \\
    \hline
    GA(NN) & value \\
    \hline
    ACO & value\\
    \hline
  \end{tabular}
\end{center}

\subsection{}

Both problems share most of the code. The only differences is in the main files
(PSO22) and the evaluation.

\subsubsection*{a}

The minimum found by the algorithm is f(5,4) = 1. This algorithm is fairly fast
so 10000 iterations does not take much time. It gives fairly accurate results
with as few as 300 iterations.

\subsubsection*{b}

The two minimums found are:
\begin{align*}
  f() &= -737 \\
  f() &= -737
\end{align*}

This function is not as lenient with the number of iterations as \emph{a} but
$10000$ tends to find one of the minimums every time.

\end{document}
