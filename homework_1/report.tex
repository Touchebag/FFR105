\documentclass{article}

\usepackage{fancyhdr}
\usepackage{mathtools}

\pagestyle{fancy}

\lhead{Robin Touche 900610-3270}
\rhead{robint@student.chalmers.se}

\begin{document}
\section{Home problem 1}
\subsection{}
\subsubsection{}

\begin{align*}
  &f_\mathit{p}(\mathbf{x}; \mu) = f(\mathbf{x}) + \mu \cdot max \left[ g(\mathbf{x}), 0 \right]  = \\
  &(x_1 - 1)^2 + 2(x_2 - 2)^2 + \mu \cdot max \left[ (x_1^2 + x_2^2 - 1 ), 0 \right]^2
\end{align*}

\subsubsection{}

Since the penalty term from the previous problem depends on whether
$g(\mathbf{x}) > 0$ or not we end up with two different cases.

\begin{align*}
  & \nabla f_\mathit{p}(\mathbf{x}; \mu) =
  \begin{cases}
    \frac{\partial}{\partial x_1}f_\mathit{p}(\mathbf{x}; \mu) \hat{i} +
    \frac{\partial}{\partial x_2}f_\mathit{p}(\mathbf{x}; \mu) \hat{j} &,g(\mathbf{x}) > 0 \\
    \frac{\partial}{\partial x_1}f(\mathbf{x}) \hat{i} +
    \frac{\partial}{\partial x_2}f(\mathbf{x}) \hat{j} &,g(\mathbf{x}) \leq 0 \\
  \end{cases} \\
  & =
  \begin{cases}
    \mu ((4 x_1^3 + x_1 (4 x_2^2-4))+2 x_1-2) \hat{i} + \\
    \hspace{1em} \mu ((4 x_1^2 x_2+x_2 (4 x_2^2-4))+4 x_2-8) \hat{j} &,\vspace{0.5em}g(\mathbf{x}) > 0 \\
    (2 (x_1 - 1)) \hat{i} + (4 (x_2 - 2)) \hat{j} &,g(\mathbf{x}) \leq 0
  \end{cases}
\end{align*}

\subsubsection{}
\begin{align*}
  Min[f(\mathbf{x})] \Rightarrow \nabla f(\mathbf{x}) = 0 \Rightarrow
  (2 (x_1 - 1)) \hat{i} + (4 (x_2 - 2)) \hat{j} = 0 \\`
  \Rightarrow
  \begin{cases}
    x_1 = 1 \\
    x_2 = 2
  \end{cases}
\end{align*}
Since we only get one solution we know that this is the global maximum or
minimum. Comparing with any other point gives that it is a minimum with the
value $0$.

\subsubsection{}
See enclosed files

\subsubsection{}
Here are the results of some different values of $\mu$.
\begin{tabular}{c | c | c}
  $\mu$ & $x_1^*$ & $x_2^*$ \\
  \hline
  1    & 0.434 & 1.210 \\
  10   & 0.331 & 0.996 \\
  100  & 0.314 & 0.955 \\
  1000 & 0.312 & 0.951
\end{tabular}

\subsection{}
\subsubsection{}

First we calculate the partial derivatives of the function.

\begin{align*}
  \begin{cases}
    f'_{x_1}(\mathbf{x}) = 8 (x_1 - x_2) \\
    f'_{x_2}(\mathbf{x}) = - x_1 + 8 x_2 - 6
  \end{cases}
\end{align*}
Setting both to $0$ and solving gives us a stationary point of:

\begin{align*}
  \begin{cases}
    8 (x_1 - x_2)     = 0 \\
    - x_1 + 8 x_2 - 6 = 0
  \end{cases}
  \Rightarrow f\left(\frac{2}{21}, \frac{16}{21} \right) \approx -2.2948
\end{align*}
Next we try the three edges where $x_2 = 1$ , $ x_1 = 0$ and $ x_1 = x_2$.

\begin{align*}
  \begin{cases}
    f'(x_1, 1)   &= 8 (x_1 - 1) \\
    f'(0, x_2)   &= 8 x_2 - 6 \\
    f'(x_1, x_1) &= 7 x_1^2 - 6x_1
  \end{cases}
  \Rightarrow
  \begin{cases}
    8 (x_1 - 1) &= 0 \\
    8 x_2 - 6   &= 0 \\
    7 x_1^2 - 6x_1  &= 0
  \end{cases}
  \Rightarrow
  \begin{cases}
    f'\left(\frac{1}{8}, 1 \right)           &= -2.0625 \\
    f'\left(0, \frac{3}{4} \right)           &= -2.25   \\
    f'\left(\frac{3}{7}, \frac{3}{7} \right) &= -1.2857
  \end{cases}
\end{align*}
Finally we have to test the three corners.

\begin{align*}
  \begin{cases}
    f(0,0) &= 0 \\
    f(0,1) &= 1 \\
    f(1,1) &= -2
  \end{cases}
\end{align*}
We're looking for a minimum so the answer is the point $\left(\frac{2}{21}, \frac{16}{21} \right)$.

\subsubsection{}

\begin{align*}
  \begin{cases}
    f(\mathbf{x}) &= 15 + 2x_1  3y \\
    h(\mathbf{x}) &= x_1^2 + x_1x_2 + x_2^2 - 21 = 0
  \end{cases}
  \Rightarrow
  L(\mathbf{x}, \lambda) = f(\mathbf{x}) + \lambda h(\mathbf{x})
\end{align*}
We calculate the partial derviatives of $L$ and set them to $0$ to get the solutions:

\begin{align*}
  \begin{cases}
    L'_{x_1} = 2 + \lambda (2x_1 + x_2) &= 0 \\
    L'_{x_2} = 3 + \lambda (x_1 + 2x_2) &= 0 \\
    L'_{\lambda} = x_1^2 + x_1 x_2 + x_2^2 - 21 &= 0
  \end{cases}
  \Rightarrow
  \begin{cases}
    f(1, 4) &= 29 \\
    f(-1, -4) &= 1 \\
  \end{cases}
\end{align*}
The point $(-1, -4)$ gives us the minimum of $1$.

\subsection{}
\subsubsection*{1.3.a}

See enclosed files.

\subsubsection*{1.3.b}

The standard parameters that was used (unless specified) are as follows: \\
\begin{tabular}{l | c}
  populationSize                & 100\\
  numberOfGenes                 & 50\\
  crossoverProbability          & 0.8\\
  mutationProbability           & 0.025\\
  tournamentSize                & 2\\
  tournamentSelectionParamenter & 0.75\\
  numberOfGenerations           & 100\\
\end{tabular}\\\\
Which resulted in both an average fitness and a standard deviation of 0.3333.

\emph{Note: when changing the population size the number of generations was changed correspondingly to have the same total number of generated individuals.}\\\\
\begin{tabular}{c | c | c}
  Population size & Average fitness & Standard Deviation\\
  \hline
  10 & 0.1989 & 0.1548\\
  50 & 0.3172 & 0.0702\\
  100 & 0.3333 & $1.5203\cdot10^{-5}$\\
  500 & 0.3161 & 0.0209\\
  1000 & 0.2846 & 0.0491\\
  \hline
  Crossover probability & Average fitness & Standard Deviation\\
  \hline
  0.2 & 0.3333 & $2.4346\cdot10^{-4}$\\
  0.4 & 0.3333 & $9.9167\cdot10^{-5}$\\
  0.6 & 0.3332 & $3.4182\cdot10^{-4}$\\
  0.8 & 0.3332 & $3.4104\cdot10^{-4}$\\
  1.0 & 0.3332 & $2.0860\cdot10^{-4}$\\
  \hline
  Mutation probability & Average fitness & Standard Deviation\\
  \hline
  0.0 & 0.1767 & 0.1151\\
  0.01 & 0.3120 & 0.0687\\
  0.025 & 0.3330 & 0.0014\\
  0.4 & 0.2472 & 0.0498\\
  0.8 & 0.2737 & 0.0473\\
  \hline
  Tournament size & Average fitness & Standard Deviation\\
  \hline
  1 & 0.3075 & 0.0511\\
  2 & 0.3332 & $1.9973\cdot10^{-4}$\\
  3 & 0.3276 & 0.0251\\
  5 & 0.3094 & 0.0570\\
  10 & 0.2843 & 0.0963\\
  \hline
  Tournament selection parameter & Average fitness & Standard Deviation\\
  \hline
  0.5 & 0.3213 & 0.0264\\
  0.6 & 0.3301 & 0.0044\\
  0.75 & 0.3333 & $4.6042\cdot10^{-5}$\\
  0.8 & 0.3333 & $3.5318\cdot10^{-5}$\\
  0.9 & 0.3333 & $4.5816\cdot10^{-6}$\\
\end{tabular}\\\\

\subsubsection*{1.3.c}



\end{document}
