\documentclass{article}

\usepackage{fancyhdr}
\usepackage{mathtools}

\pagestyle{fancy}

\lhead{Robin Touche 900610-3270}
\rhead{robint@student.chalmers.se}

\begin{document}
\section{Home problem 1}
\subsection{}
\subsubsection{}

\begin{align*}
  &f_\mathit{p}(\mathbf{x}; \mu) = f(\mathbf{x}) + \mu \cdot max \left[ g(\mathbf{x}), 0 \right]  = \\
  &(x_1 - 1)^2 + 2(x_2 - 2)^2 + \mu \cdot max \left[ (x_1^2 + x_2^2 - 1 ), 0 \right]
\end{align*}

\subsubsection{}

Since the penalty term from the previous problem depends on whether
$g(\mathbf{x}) > 0$ or not we end up with two different cases.

\begin{align*}
  & \nabla f_\mathit{p}(\mathbf{x}; \mu) =
  \begin{cases}
    \frac{\partial f}{\partial x_1}f_\mathit{p}(\mathbf{x}; \mu) \hat{i} +
    \frac{\partial f}{\partial x_2}f_\mathit{p}(\mathbf{x}; \mu) \hat{j} &,g(\mathbf{x}) > 0 \\
    \frac{\partial f}{\partial x_1}f(\mathbf{x}) \hat{i} +
    \frac{\partial f}{\partial x_2}f(\mathbf{x}) \hat{j} &,g(\mathbf{x}) \leq 0 \\
  \end{cases} \\
  & =
  \begin{cases}
    ((2\mu + 2)x_1 - 2) \hat{i} + (2(\mu + 2)x_2 - 8) \hat{j} &,g(\mathbf{x}) > 0 \\
    (2 (x_1 - 1)) \hat{i} + (2 (x_2 - 4)) \hat{j} &,g(\mathbf{x}) \leq 0
  \end{cases}
\end{align*}

\subsection{}
\subsubsection{}

First we take calculate the partial derivatives of the function.

\begin{align*}
  \begin{cases}
    f'_{x_1}(\mathbf{x}) = 8 (x_1 - x_2) \\
    f'_{x_2}(\mathbf{x}) = - x_1 + 8 x_2 - 6
  \end{cases}
\end{align*}
Setting both to $0$ and solving gives us a stationary point of:

\begin{align*}
  \begin{cases}
    8 (x_1 - x_2)     = 0 \\
    - x_1 + 8 x_2 - 6 = 0
  \end{cases}
  \Rightarrow f\left(\frac{3}{31}, \frac{24}{31} \right) = -2.2851
\end{align*}
Next we try the three edges where $x_2 = 1$ , $ x_1 = 0$ and $ x_1 = x_2$.

\begin{align*}
  \begin{cases}
    f'(x_1, 1)   &= 8 (x_1 - 1) \\
    f'(0, x_2)   &= 8 x_2 - 6 \\
    f'(x_1, x_1) &= 7 x^2 - 6x
  \end{cases}
  \Rightarrow
  \begin{cases}
    8 (x_1 - 1) &= 0 \\
    8 x_2 - 6   &= 0 \\
    7 x^2 - 6x  &= 0
  \end{cases}
  \Rightarrow
  \begin{cases}
    f'\left(\frac{1}{8}, 1 \right)           &= -2.0625 \\
    f'\left(0, \frac{3}{4} \right)           &= -2.25   \\
    f'\left(\frac{3}{7}, \frac{3}{7} \right) &= -1.2857
  \end{cases}
\end{align*}
Finally we have to test the three corners.

\begin{align*}
  \begin{cases}
    f(0,0) &= 0 \\
    f(0,1) &= 1 \\
    f(1,1) &= -2
  \end{cases}
\end{align*}
We're looking for a minimum so the answer is the point $\left(\frac{3}{31}, \frac{24}{31} \right)$.

\subsubsection{}

\begin{align*}
  \begin{cases}
    f(\mathbf{x}) &= 15 + 2x_1  3y \\
    h(\mathbf{x}) &= x_1^2 + x_1y_2 + y_2^2 - 21 = 0
  \end{cases}
  \Rightarrow
  L(\mathbf{x}, \lambda) = f(\mathbf{x}) + \lambda h(\mathbf{x})
\end{align*}
We calculate the partial derviatives of $L$ and set them to $0$ to get the solutions:

\begin{align*}
  \begin{cases}
    L'_{x_1} = 2 + \lambda (2x_1 + x_2) &= 0 \\
    L'_{x_2} = 3 + \lambda (x_1 + 2x_2) &= 0 \\
    L'_{\lambda} = x_1^2 + x_1 x_2 + x_2^2 - 21 &= 0
  \end{cases}
  \Rightarrow
  \begin{cases}
    f(1, 4) &= 29 \\
    f(-1, -4) &= 1 \\
  \end{cases}
\end{align*}
The point $(-1, -4)$ gives us the minimum of $1$.

\subsection{}
\subsubsection{}

See enclosed files.

\subsubsection{}



\end{document}
